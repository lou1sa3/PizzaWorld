\documentclass[11pt,a4paper]{article}
\usepackage[utf8]{inputenc}
\usepackage[T1]{fontenc}
\usepackage{geometry}
\usepackage{hyperref}
\usepackage{listings}
\usepackage{xcolor}
\usepackage{booktabs}
\usepackage{longtable}
\usepackage{fancyhdr}
\usepackage{titlesec}
\usepackage{enumitem}
\usepackage{graphicx}
\usepackage{amssymb}

% Page setup
\geometry{top=2.5cm, bottom=2.5cm, left=2.5cm, right=2.5cm}

% Colors
\definecolor{codeblue}{RGB}{0,100,200}
\definecolor{codegray}{RGB}{128,128,128}
\definecolor{codegreen}{RGB}{0,128,0}
\definecolor{green}{RGB}{0,128,0}
\definecolor{orange}{RGB}{255,165,0}
\definecolor{red}{RGB}{255,0,0}

% Code styling
\lstset{
    basicstyle=\ttfamily\footnotesize,
    backgroundcolor=\color{gray!5},
    frame=single,
    breaklines=true,
    showstringspaces=false,
    keywordstyle=\color{codeblue},
    commentstyle=\color{codegreen},
    stringstyle=\color{red},
    numbers=left,
    numberstyle=\tiny\color{codegray}
}

% Header/Footer
\pagestyle{fancy}
\fancyhf{}
\fancyhead[L]{PizzaWorld Technical Documentation}
\fancyhead[R]{v1.0}
\fancyfoot[C]{\thepage}

% Hyperlinks
\hypersetup{
    colorlinks=true,
    linkcolor=blue,
    urlcolor=blue
}

\begin{document}

\begin{titlepage}
    \centering
    \vspace*{2cm}
    
    {\Huge\bfseries PizzaWorld Dashboard}\\[0.5cm]
    {\Large Technical Documentation}
    
    \vspace{2cm}
    
    {\large Complete Developer Guide \& Code Documentation}
    
    \vspace{3cm}
    
    \begin{tabular}{rl}
        \textbf{Project:} & PizzaWorld Business Intelligence Dashboard \\[0.2cm]
        \textbf{Version:} & 1.0 \\[0.2cm]
        \textbf{Framework:} & Spring Boot 3.4.6 + Angular 19 \\[0.2cm]
        \textbf{Database:} & Supabase PostgreSQL \\[0.2cm]
        \textbf{AI Integration:} & Google Gemma AI \\[0.2cm]
        \textbf{Repository:} & \href{https://github.com/luigids03/PizzaWorld}{github.com/luigids03/PizzaWorld}
    \end{tabular}
    
    \vfill
    
    {\large \textcopyright{} 2025 - Academic Project Documentation}
\end{titlepage}

\tableofcontents
\newpage

\section{System Architecture}

\subsection{Overview}
PizzaWorld is a full-stack business intelligence dashboard with microservices architecture:

\begin{itemize}[leftmargin=*]
    \item \textbf{Backend}: Spring Boot REST API with 90+ endpoints
    \item \textbf{Frontend}: Angular SPA with TypeScript
    \item \textbf{AI Layer}: Google Gemma integration with knowledge base
    \item \textbf{Security}: JWT authentication with role-based access
    \item \textbf{Database}: Supabase PostgreSQL with 15+ materialized views
    \item \textbf{Analytics}: Advanced store, customer, and product analytics
    \item \textbf{Export System}: Comprehensive CSV export for all data views
    \item \textbf{Email Support}: Background email processing system
\end{itemize}

\subsection{Technology Stack}

\begin{longtable}{|l|l|l|}
\hline
\textbf{Layer} & \textbf{Technology} & \textbf{Version} \\
\hline
\endhead
Backend Framework & Spring Boot & 3.4.6 \\
\hline
Security & Spring Security & 6.x \\
\hline
Database & PostgreSQL (Supabase) & Latest \\
\hline
ORM & Spring Data JPA & 3.x \\
\hline
Frontend Framework & Angular & 19 \\
\hline
Language & TypeScript & 5.7 \\
\hline
UI Library & PrimeNG + Tailwind & 19 + 3.4 \\
\hline
Charts & ApexCharts & 3.41 \\
\hline
AI Integration & Google Gemma & Latest \\
\hline
Build Tool & Maven + Angular CLI & Latest \\
\hline
\end{longtable}

\section{Backend Architecture}

\subsection{Package Structure}

\begin{lstlisting}[language=text, caption=Backend Package Organization]
src/main/java/pizzaworld/
├── controller/          # REST API Controllers
│   ├── AuthController.java
│   ├── OptimizedPizzaController.java
│   ├── AIController.java
│   └── SupportController.java
├── service/            # Business Logic Layer
│   ├── OptimizedPizzaService.java
│   ├── AIService.java
│   ├── GemmaAIService.java
│   ├── StaticDocRetriever.java
│   ├── UserService.java
│   └── EmailService.java
├── repository/         # Data Access Layer
│   ├── OptimizedPizzaRepo.java
│   └── UserRepo.java
├── model/             # Entity Classes
│   ├── User.java
│   ├── ChatMessage.java
│   ├── AIInsight.java
│   └── CustomUserDetails.java
├── security/          # Security Components
│   ├── JwtAuthFilter.java
│   └── SecurityConfig.java
├── config/           # Configuration Classes
│   ├── CorsConfig.java
│   ├── EmailConfig.java
│   └── PizzaConfig.java
├── dto/              # Data Transfer Objects
│   ├── DashboardKpiDto.java
│   ├── ConsolidatedDto.java
│   └── SalesKpiDto.java
└── util/             # Utility Classes
    ├── JwtUtil.java
    ├── CsvExportUtil.java
    └── BcryptTool.java
\end{lstlisting}

\subsection{Core Controllers}

\subsubsection{AuthController}
Handles user authentication and authorization:

\begin{lstlisting}[language=java, caption=Authentication Endpoints]
@RestController
@RequestMapping("/api")
public class AuthController {
    
    @PostMapping("/login")
    public ResponseEntity<?> login(@RequestBody LoginRequest request)
    
    @GetMapping("/me") 
    public ResponseEntity<?> getCurrentUser(@AuthenticationPrincipal CustomUserDetails user)
    
    @PostMapping("/logout")
    public ResponseEntity<?> logout()
    
    @PostMapping("/create-test-user")
    public ResponseEntity<?> createTestUser()
}
\end{lstlisting}

\subsubsection{OptimizedPizzaController}
Main business logic controller for KPIs and analytics:

\begin{lstlisting}[language=java, caption=Business Endpoints]
@RestController 
@RequestMapping("/api/v2")
public class OptimizedPizzaController {
    
    @GetMapping("/dashboard/kpis")
    public ResponseEntity<DashboardKpiDto> getDashboardKPIs(@AuthenticationPrincipal CustomUserDetails user)
    
    @GetMapping("/dashboard/kpis/export")
    public void exportDashboardKPIs(@AuthenticationPrincipal CustomUserDetails user, 
                                   HttpServletResponse response)
    
    @GetMapping("/orders/recent")
    public ResponseEntity<List<Map<String, Object>>> getRecentOrders(
        @RequestParam(defaultValue = "50") int limit,
        @AuthenticationPrincipal CustomUserDetails user)
    
    @GetMapping("/analytics/revenue/by-year")
    public ResponseEntity<List<Map<String, Object>>> getRevenueByYear(
        @AuthenticationPrincipal CustomUserDetails user)
    
    @GetMapping("/analytics/customer-lifetime-value")
    public ResponseEntity<List<Map<String, Object>>> getCustomerLifetimeValue(
        @RequestParam(defaultValue = "100") Integer limit,
        @RequestParam(required = false) List<String> states,
        @RequestParam(required = false) List<String> storeIds,
        @AuthenticationPrincipal CustomUserDetails user)
    
    @GetMapping("/stores/{storeId}/analytics/overview")
    public ResponseEntity<Map<String, Object>> getStoreAnalyticsOverview(
        @PathVariable String storeId,
        @RequestParam(required = false) String timePeriod,
        @AuthenticationPrincipal CustomUserDetails user)
    
    @GetMapping("/products/kpi")
    public ResponseEntity<Map<String, Object>> getProductKPI(
        @RequestParam String sku,
        @RequestParam(defaultValue = "all-time") String timePeriod,
        @AuthenticationPrincipal CustomUserDetails user)
    
    @GetMapping("/kpis/global-store")
    public ResponseEntity<List<Map<String, Object>>> getGlobalStoreKPIs(
        @AuthenticationPrincipal CustomUserDetails user)
    
    @GetMapping("/analytics/store-capacity-v3/summary")
    public ResponseEntity<List<Map<String, Object>>> getStoreCapacityV3Summary(
        @RequestParam(required = false) List<String> states,
        @RequestParam(required = false) List<String> storeIds,
        @AuthenticationPrincipal CustomUserDetails user)
    
    @PostMapping("/stores/{storeId}/analytics/compare")
    public ResponseEntity<List<Map<String, Object>>> getStoreComparePeriods(
        @PathVariable String storeId,
        @RequestBody Map<String, Object> requestBody,
        @AuthenticationPrincipal CustomUserDetails user)
}
\end{lstlisting}

\subsubsection{AIController}
AI assistant integration with Google Gemma:

\begin{lstlisting}[language=java, caption=AI Endpoints]
@RestController
@RequestMapping("/api/ai")
public class AIController {
    
    @PostMapping("/chat")
    public ResponseEntity<?> chat(@RequestBody ChatRequest request, 
                                 @AuthenticationPrincipal CustomUserDetails user)
    
    @PostMapping(path = "/chat/stream", produces = MediaType.TEXT_EVENT_STREAM_VALUE)
    public Flux<ServerSentEvent<String>> chatStream(@RequestBody ChatRequest request,
                                                   @AuthenticationPrincipal CustomUserDetails user)
    
    @GetMapping("/chat/history/{sessionId}")
    public ResponseEntity<?> getChatHistory(@PathVariable String sessionId,
                                           @AuthenticationPrincipal CustomUserDetails user)
    
    @PostMapping("/analyze")
    public ResponseEntity<?> analyzeQuery(@RequestBody AnalyzeRequest request,
                                        @AuthenticationPrincipal CustomUserDetails user)
    
    @GetMapping("/insights")
    public ResponseEntity<?> getInsights(@AuthenticationPrincipal CustomUserDetails user)
    
    @GetMapping("/health")
    public ResponseEntity<?> healthCheck()
    
    @GetMapping("/config")
    public ResponseEntity<?> getPublicConfig()
    
    @GetMapping("/status")
    public ResponseEntity<?> getAIStatus(@AuthenticationPrincipal CustomUserDetails user)
    
    @PostMapping("/test")
    public ResponseEntity<?> testGoogleAI(@AuthenticationPrincipal CustomUserDetails user)
}
\end{lstlisting}

\subsubsection{SupportController}
Email support system for customer service:

\begin{lstlisting}[language=java, caption=Support Endpoints]
@RestController
@RequestMapping("/api")
public class SupportController {
    
    @Autowired
    private EmailService emailService;
    
    @PostMapping("/send-support-email")
    public ResponseEntity<?> sendSupportEmail(@RequestBody EmailRequest emailRequest) {
        // Background email processing for improved response times
        // Immediate success response - don't wait for email processing
        
        new Thread(() -> {
            try {
                emailService.sendSupportEmail(
                    emailRequest.from,
                    emailRequest.senderName,
                    emailRequest.subject,
                    emailRequest.message
                );
            } catch (Exception emailEx) {
                logger.warning("Background email failed: " + emailEx.getMessage());
            }
        }).start();
        
        return ResponseEntity.ok(Map.of(
            "success", true,
            "message", "Your message has been sent successfully!"
        ));
    }
    
    public static class EmailRequest {
        public String to;
        public String from;
        public String senderName;
        public String subject;
        public String message;
    }
}
\end{lstlisting}

\subsection{Service Layer}

\subsubsection{AIService}
Core AI orchestration service:

\begin{lstlisting}[language=java, caption=AI Service Implementation]
@Service
public class AIService {
    
    @Autowired
    private GemmaAIService gemmaAIService;
    
    @Autowired 
    private StaticDocRetriever docRetriever;
    
    // Non-persistent chat sessions (max 20 messages)
    private final Map<String, Deque<ChatMessage>> chatSessions = new ConcurrentHashMap<>();
    
    public ChatMessage processChatMessage(String sessionId, String message, User user) {
        // 1. Categorize message (support, analytics, general)
        // 2. Build conversation context 
        // 3. Retrieve knowledge snippets
        // 4. Generate AI response with Gemma
        // 5. Apply business consistency validation
        // 6. Store in session (non-persistent)
    }
    
    public List<AIInsight> generateBusinessInsights(User user) {
        // Generate role-specific insights based on user permissions
    }
}
\end{lstlisting}

\subsubsection{GemmaAIService}
Google AI integration service:

\begin{lstlisting}[language=java, caption=Google AI Integration]
@Service
public class GemmaAIService {
    
    @Value("${google.ai.api.key}")
    private String apiKey;
    
    @Value("${google.ai.model:gemma-3n-e2b-it}")
    private String model;
    
    private final WebClient webClient;
    
    public String generateResponse(String userMessage, User user, 
                                 String category, Map<String, Object> businessContext) {
        // 1. Build business-specific prompt
        // 2. Call Google AI API
        // 3. Clean and validate response
        // 4. Return contextual answer
    }
}
\end{lstlisting}

\subsection{Security Implementation}

\subsubsection{JWT Authentication}

\begin{lstlisting}[language=java, caption=JWT Security Filter]
@Component
public class JwtAuthFilter extends OncePerRequestFilter {
    
    @Override
    protected void doFilterInternal(HttpServletRequest request,
                                  HttpServletResponse response,
                                  FilterChain filterChain) {
        // 1. Extract JWT token from Authorization header
        // 2. Validate token signature and expiration
        // 3. Load user details and permissions
        // 4. Set security context
    }
}
\end{lstlisting}

\subsubsection{Security Configuration}

\begin{lstlisting}[language=java, caption=Security Configuration]
@Configuration
@EnableWebSecurity
public class SecurityConfig {
    
    @Bean
    public SecurityFilterChain filterChain(HttpSecurity http) throws Exception {
        return http
            .csrf(csrf -> csrf.disable())
            .sessionManagement(session -> session.sessionCreationPolicy(STATELESS))
            .authorizeHttpRequests(auth -> auth
                .requestMatchers("/api/login", "/api/health").permitAll()
                .requestMatchers("/api/admin/**").hasRole("HQ_ADMIN")
                .anyRequest().authenticated()
            )
            .addFilterBefore(jwtAuthFilter, UsernamePasswordAuthenticationFilter.class)
            .build();
    }
}
\end{lstlisting}

\subsection{Database Layer}

\subsubsection{Materialized Views}
Performance-optimized views for analytics:

\begin{lstlisting}[language=sql, caption=Key Materialized Views]
-- KPI Views (Role-based)
kpis_global_hq          -- Company-wide KPIs
kpis_global_state       -- State-level KPIs  
kpis_global_store       -- Store-level KPIs

-- Revenue Analytics
revenue_by_year_hq      -- Annual revenue trends
revenue_by_month_hq     -- Monthly analysis
revenue_by_week_hq      -- Weekly patterns
revenue_by_day_hq       -- Daily tracking
revenue_by_hour_hq      -- Hourly analysis

-- Performance Views
store_performance_hq    -- Store ranking
customer_lifetime_value -- CLV analysis
customer_retention_analysis -- Cohort analysis
top_products_hq        -- Product performance
\end{lstlisting}

\subsubsection{Repository Layer}

\begin{lstlisting}[language=java, caption=Optimized Repository]
@Repository
public class OptimizedPizzaRepo {
    
    @Autowired
    private JdbcTemplate jdbcTemplate;
    
    // Role-based data access with native SQL
    public DashboardKpiDto getDashboardKpis(User user) {
        String sql = switch(user.getRole()) {
            case "HQ_ADMIN" -> "SELECT * FROM kpis_global_hq";
            case "STATE_MANAGER" -> "SELECT * FROM kpis_global_state WHERE state_abbr = ?";
            case "STORE_MANAGER" -> "SELECT * FROM kpis_global_store WHERE storeid = ?";
            default -> throw new IllegalArgumentException("Invalid role: " + user.getRole());
        };
        
        return jdbcTemplate.queryForObject(sql, 
            new BeanPropertyRowMapper<>(DashboardKpiDto.class),
            getParametersForRole(user));
    }
    
    public List<Map<String, Object>> getStoreAnalyticsOverview(User user, String storeId, 
                                                              String timePeriod) {
        String sql = buildStoreAnalyticsQuery(user.getRole(), timePeriod);
        return jdbcTemplate.queryForList(sql, storeId);
    }
    
    public List<Map<String, Object>> getCustomerLifetimeValue(User user, Integer limit, 
                                                             List<String> states) {
        String sql = buildCustomerCLVQuery(user.getRole());
        return jdbcTemplate.queryForList(sql, getRoleSpecificParameters(user));
    }
    
    public List<Map<String, Object>> getProductPerformanceAnalytics(User user, 
                                                                   String category, 
                                                                   Integer limit) {
        String sql = buildProductAnalyticsQuery(user.getRole(), category);
        return jdbcTemplate.queryForList(sql, limit);
    }
    
    private Object[] getParametersForRole(User user) {
        return switch(user.getRole()) {
            case "STATE_MANAGER" -> new Object[]{user.getStateAbbr()};
            case "STORE_MANAGER" -> new Object[]{user.getStoreId()};
            default -> new Object[]{};
        };
    }
}
\end{lstlisting}

\section{Frontend Architecture}

\subsection{Angular Structure}

\begin{lstlisting}[language=text, caption=Frontend Structure]
src/app/
├── pages/                    # Feature Modules
│   ├── dashboard/           # Main dashboard
│   ├── orders/             # Order management
│   ├── products/           # Product catalog
│   ├── stores/             # Store management
│   ├── customer-analytics/ # Customer insights
│   ├── delivery-metrics/   # Delivery tracking
│   ├── profile/            # User profile
│   ├── contact-support/    # Support system
│   └── login/              # Authentication
├── core/                   # Core Services
│   ├── auth.service.ts     # Authentication
│   ├── ai.service.ts       # AI integration
│   ├── kpi.service.ts      # KPI management
│   ├── auth.guard.ts       # Route protection
│   ├── cache.service.ts    # Client caching
│   ├── notification.service.ts # Notifications
│   └── theme.service.ts    # Theme management
└── shared/                 # Reusable Components
    ├── ai-chatbot/         # AI chat interface
    ├── loading-popup/      # Loading indicators
    ├── notification/       # Alert system
    └── sidebar/            # Navigation
\end{lstlisting}

\subsection{Core Services}

\subsubsection{AuthService}

\begin{lstlisting}[language=typescript, caption=Authentication Service]
@Injectable({providedIn: 'root'})
export class AuthService {
  private currentUserSubject = new BehaviorSubject<User | null>(null);
  public currentUser$ = this.currentUserSubject.asObservable();

  login(credentials: LoginRequest): Observable<LoginResponse> {
    return this.http.post<LoginResponse>('/api/login', credentials)
      .pipe(
        tap(response => {
          localStorage.setItem('token', response.token);
          this.currentUserSubject.next(response.user);
        })
      );
  }

  hasRole(role: string): boolean {
    const user = this.currentUserSubject.value;
    return user?.role === role;
  }
}
\end{lstlisting}

\subsubsection{AIService}

\begin{lstlisting}[language=typescript, caption=AI Service]
@Injectable({providedIn: 'root'})
export class AIService {
  private chatHistorySubject = new BehaviorSubject<ChatMessage[]>([]);
  public chatHistory$ = this.chatHistorySubject.asObservable();

  sendMessage(message: string, context?: string): Observable<ChatMessage> {
    return this.http.post<ChatResponse>('/api/ai/chat', {
      sessionId: this.currentSessionId,
      message,
      context
    });
  }

  sendMessageStream(message: string, context?: string): Observable<string> {
    // Handle Server-Sent Events for real-time streaming
    return new Observable<string>((observer) => {
      // Fetch-based streaming implementation
    });
  }

  getChatHistory(): Observable<ChatMessage[]> {
    return this.http.get<any>('/api/ai/chat/history/' + this.currentSessionId);
  }

  getInsights(): Observable<AIInsight[]> {
    return this.http.get<InsightsResponse>('/api/ai/insights');
  }

  analyzeQuery(query: string, context?: string, type?: string): Observable<AnalysisResponse> {
    return this.http.post<AnalysisResponse>('/api/ai/analyze', { query, context, type });
  }

  healthCheck(): Observable<boolean> {
    return this.http.get<any>('/api/ai/health');
  }

  testGoogleAI(): Observable<any> {
    return this.http.post<any>('/api/ai/test', {});
  }

  getAIStatus(): Observable<any> {
    return this.http.get<any>('/api/ai/status');
  }

  clearChatSession(): void {
    this.currentSessionId = this.generateSessionId();
    this.chatHistorySubject.next([]);
  }
}
\end{lstlisting}

\subsection{Component Architecture}

\subsubsection{Dashboard Component}

\begin{lstlisting}[language=typescript, caption=Dashboard Implementation]
@Component({
  selector: 'app-dashboard',
  templateUrl: './dashboard.component.html',
  changeDetection: ChangeDetectionStrategy.OnPush
})
export class DashboardComponent implements OnInit {
  
  kpis$ = this.kpiService.getDashboardKpis();
  recentOrders$ = this.kpiService.getRecentOrders();
  revenueChart$ = this.kpiService.getRevenueTrend();

  constructor(
    private kpiService: KpiService,
    private authService: AuthService,
    private cdr: ChangeDetectorRef
  ) {}

  ngOnInit() {
    // Auto-refresh every 30 seconds
    interval(30000).pipe(
      startWith(0),
      switchMap(() => this.loadDashboardData())
    ).subscribe();
  }

  private loadDashboardData() {
    return combineLatest([
      this.kpis$,
      this.recentOrders$,
      this.revenueChart$
    ]);
  }
}
\end{lstlisting}

\subsubsection{AI Chatbot Component}

\begin{lstlisting}[language=typescript, caption=AI Chatbot]
@Component({
  selector: 'app-ai-chatbot',
  templateUrl: './ai-chatbot.component.html'
})
export class AIChatbotComponent {
  
  messages$ = this.aiService.chatHistory$;
  isStreaming = false;

  sendMessage(message: string) {
    this.isStreaming = true;
    
    this.aiService.streamChat(message).subscribe({
      next: (token) => this.appendStreamingToken(token),
      complete: () => this.isStreaming = false,
      error: (error) => this.handleStreamError(error)
    });
  }

  private appendStreamingToken(token: string) {
    // Real-time token streaming for smooth UX
  }
}
\end{lstlisting}

\section{AI Integration}

\subsection{Google Gemma Integration}

\subsubsection{AI Configuration}

\begin{lstlisting}[language=properties, caption=AI Configuration]
# Google AI Settings
google.ai.api.key=${GOOGLE_AI_API_KEY}
google.ai.model=${GOOGLE_AI_MODEL:gemma-3n-e2b-it}
google.ai.enabled=${GOOGLE_AI_ENABLED:true}
\end{lstlisting}

\subsubsection{Knowledge Base System}

\begin{lstlisting}[language=java, caption=Document Retrieval]
@Service
public class StaticDocRetriever {
    
    private List<KnowledgeChunk> knowledgeBase;
    
    @PostConstruct
    public void loadKnowledgeBase() {
        // Load from resources/knowledge/
        // - business-operations.md
        // - technical-guide.md  
        // - faq.md
    }
    
    public Optional<String> findMatch(String query) {
        return knowledgeBase.stream()
            .filter(chunk -> matchesKeywords(chunk, query))
            .findFirst()
            .map(KnowledgeChunk::getContent);
    }
}
\end{lstlisting}

\subsection{Chat System Features}

\begin{itemize}[leftmargin=*]
    \item \textbf{Non-persistent Sessions}: Last 20 messages per session
    \item \textbf{Real-time Streaming}: Server-Sent Events for token-by-token delivery
    \item \textbf{Business Context}: AI responses include live business data
    \item \textbf{Role-based Responses}: Tailored to user permissions
    \item \textbf{Intelligent Fallbacks}: Rule-based responses when AI unavailable
    \item \textbf{Knowledge Integration}: Contextual responses using knowledge base
\end{itemize}

\section{Database Schema}

\subsection{Core Tables}

\begin{lstlisting}[language=sql, caption=Core Database Tables]
-- User Management
CREATE TABLE users (
    id SERIAL PRIMARY KEY,
    username VARCHAR(255) UNIQUE NOT NULL,
    password_hash VARCHAR(255) NOT NULL,
    role VARCHAR(50) NOT NULL,
    state_abbr VARCHAR(10),
    store_id INTEGER,
    created_at TIMESTAMP DEFAULT CURRENT_TIMESTAMP
);

-- Orders and Items
CREATE TABLE orders (
    id SERIAL PRIMARY KEY,
    customer_id INTEGER,
    store_id INTEGER,
    order_date TIMESTAMP,
    total_amount DECIMAL(10,2),
    status VARCHAR(50)
);

CREATE TABLE order_items (
    id SERIAL PRIMARY KEY,
    order_id INTEGER REFERENCES orders(id),
    product_sku VARCHAR(100),
    quantity INTEGER,
    unit_price DECIMAL(10,2)
);

-- Products and Stores
CREATE TABLE products (
    sku VARCHAR(100) PRIMARY KEY,
    product_name VARCHAR(255),
    category VARCHAR(100),
    size VARCHAR(50),
    price DECIMAL(10,2),
    launch_date DATE
);

CREATE TABLE stores (
    id SERIAL PRIMARY KEY,
    store_name VARCHAR(255),
    city VARCHAR(100),
    state_name VARCHAR(100),
    state_abbr VARCHAR(10),
    zip_code VARCHAR(20)
);
\end{lstlisting}

\subsection{Role-Based Data Access}

\begin{lstlisting}[language=sql, caption=Role-Based Views]
-- HQ Admin: All data
CREATE MATERIALIZED VIEW kpis_global_hq AS
SELECT 
    SUM(total_amount) as total_revenue,
    COUNT(*) as total_orders,
    AVG(total_amount) as avg_order_value,
    COUNT(DISTINCT customer_id) as total_customers
FROM orders;

-- State Manager: State-filtered data  
CREATE MATERIALIZED VIEW kpis_global_state AS
SELECT 
    s.state_abbr,
    SUM(o.total_amount) as total_revenue,
    COUNT(o.*) as total_orders,
    AVG(o.total_amount) as avg_order_value
FROM orders o
JOIN stores s ON o.store_id = s.id
GROUP BY s.state_abbr;

-- Store Manager: Store-specific data
CREATE MATERIALIZED VIEW kpis_global_store AS  
SELECT 
    store_id,
    SUM(total_amount) as total_revenue,
    COUNT(*) as total_orders,
    AVG(total_amount) as avg_order_value
FROM orders
GROUP BY store_id;
\end{lstlisting}

\section{Configuration Management}

\subsection{Environment Variables}

\begin{lstlisting}[language=properties, caption=Application Configuration]
# Database Configuration
spring.datasource.url=${DB_URL}
spring.datasource.username=${DB_USERNAME}
spring.datasource.password=${DB_PASSWORD}

# Security
jwt.secret=${JWT_SECRET}

# Email Configuration  
spring.mail.username=pizzaworldplus@gmail.com
spring.mail.password=${GMAIL_APP_PASSWORD}

# AI Configuration
google.ai.api.key=${GOOGLE_AI_API_KEY}
google.ai.model=${GOOGLE_AI_MODEL:gemma-3n-e2b-it}

# Performance Settings
spring.datasource.hikari.maximum-pool-size=30
spring.datasource.hikari.minimum-idle=10
\end{lstlisting}

\subsection{CORS Configuration}

\begin{lstlisting}[language=java, caption=CORS Setup]
@Configuration
public class CorsConfig {
    
    @Bean
    public CorsConfigurationSource corsConfigurationSource() {
        CorsConfiguration configuration = new CorsConfiguration();
        configuration.setAllowedOriginPatterns(Arrays.asList(
            "http://localhost:*",
            "https://*.onrender.com", 
            "https://pizzaworldplus.tech",
            "https://*.pizzaworldplus.tech"
        ));
        configuration.setAllowedMethods(Arrays.asList("GET", "POST", "PUT", "DELETE"));
        configuration.setAllowCredentials(true);
        return source;
    }
}
\end{lstlisting}

\section{Deployment & DevOps}

\subsection{Docker Configuration}

\begin{lstlisting}[language=dockerfile, caption=Production Dockerfile]
FROM eclipse-temurin:17-jdk-jammy
WORKDIR /app

# Copy and build application
COPY . .
RUN ./mvnw clean package -DskipTests

# Runtime configuration
EXPOSE 8080
CMD ["java", "-Xmx1400m", "-Xms512m", "-XX:+UseG1GC", "-jar", "target/*.jar"]
\end{lstlisting}

\subsection{Cloud Deployment}

\begin{lstlisting}[language=yaml, caption=Render.com Configuration]
services:
  - type: web
    name: pizzaworld-backend
    env: docker
    dockerfilePath: ./Dockerfile
    envVars:
      - key: SPRING_PROFILES_ACTIVE
        value: production
      - key: DB_URL
        fromDatabase:
          name: pizzaworld-db
          property: connectionString
          
  - type: web  
    name: pizzaworld-frontend
    runtime: static
    buildCommand: npm install && npm run build:prod
    staticPublishPath: ./dist/frontend
\end{lstlisting}

\section{Testing Strategy}

\subsection{Backend Testing}

\begin{lstlisting}[language=java, caption=Unit Test Example]
@ExtendWith(MockitoExtension.class)
class AIServiceTest {
    
    @Mock
    private GemmaAIService gemmaAIService;
    
    @Mock
    private StaticDocRetriever docRetriever;
    
    @InjectMocks
    private AIService aiService;
    
    @Test
    void testChatMessageProcessing() {
        // Given
        User user = new User("testuser", "HQ_ADMIN");
        String message = "What is our revenue?";
        
        // When
        ChatMessage response = aiService.processChatMessage("session1", message, user);
        
        // Then
        assertThat(response.getMessage()).isNotNull();
        assertThat(response.getCategory()).isEqualTo("analytics");
    }
}
\end{lstlisting}

\subsection{Frontend Testing}

\begin{lstlisting}[language=typescript, caption=Component Test]
describe('DashboardComponent', () => {
  let component: DashboardComponent;
  let fixture: ComponentFixture<DashboardComponent>;
  let kpiService: jasmine.SpyObj<KpiService>;

  beforeEach(() => {
    const spy = jasmine.createSpyObj('KpiService', ['getDashboardKpis']);
    
    TestBed.configureTestingModule({
      declarations: [DashboardComponent],
      providers: [
        { provide: KpiService, useValue: spy }
      ]
    });
    
    fixture = TestBed.createComponent(DashboardComponent);
    component = fixture.componentInstance;
    kpiService = TestBed.inject(KpiService) as jasmine.SpyObj<KpiService>;
  });

  it('should load dashboard data on init', () => {
    kpiService.getDashboardKpis.and.returnValue(of(mockKpis));
    
    component.ngOnInit();
    
    expect(kpiService.getDashboardKpis).toHaveBeenCalled();
  });
});
\end{lstlisting}

\section{Performance Optimization}

\subsection{Backend Optimizations}

\begin{itemize}[leftmargin=*]
    \item \textbf{Materialized Views}: Pre-computed analytics for instant response
    \item \textbf{Connection Pooling}: HikariCP with 30 max connections
    \item \textbf{JVM Tuning}: G1GC with 2GB heap, string deduplication
    \item \textbf{Query Optimization}: Native SQL with proper indexing
    \item \textbf{Caching}: Application-level caching for business context
\end{itemize}

\subsection{Frontend Optimizations}

\begin{itemize}[leftmargin=*]
    \item \textbf{Lazy Loading}: Route-based code splitting
    \item \textbf{OnPush Strategy}: Optimized change detection
    \item \textbf{Virtual Scrolling}: For large data sets
    \item \textbf{HTTP Caching}: Response caching with interceptors
    \item \textbf{Bundle Optimization}: Tree shaking and minification
\end{itemize}

\section{Security Implementation}

\subsection{Authentication Flow}

\begin{enumerate}[leftmargin=*]
    \item User submits credentials to \texttt{/api/login}
    \item Server validates credentials against BCrypt hash
    \item JWT token generated with user role and permissions
    \item Token returned to client with 24-hour expiration
    \item Client includes token in Authorization header
    \item \texttt{JwtAuthFilter} validates token on each request
    \item Security context set with user details and roles
\end{enumerate}

\subsection{Authorization Matrix}

\textbf{Legend:} F=Full, L=Filtered, R=Read-only, X=Denied

\begin{center}
\footnotesize
\begin{longtable}{|l|c|c|c|}
\hline
\textbf{Endpoint} & \textbf{HQ} & \textbf{ST} & \textbf{SH} \\
\hline
\endhead
dashboard/kpis & F & L & L \\
\hline
orders & F & L & L \\
\hline
products & F & L & R \\
\hline
stores & F & L & L \\
\hline
stores/analytics/** & F & L & L \\
\hline
analytics/customer-** & F & L & L \\
\hline
analytics/capacity-v3/** & F & L & L \\
\hline
analytics/peak-hours & F & F & X \\
\hline
kpis/global-store & F & L & L \\
\hline
products/kpi & F & L & L \\
\hline
products/trend & F & L & L \\
\hline
chart/time-periods/** & F & F & F \\
\hline
**/compare & F & L & L \\
\hline
**/export & F & L & L \\
\hline
ai/** & F & L & L \\
\hline
send-support-email & F & F & F \\
\hline
create-test-user & F & X & X \\
\hline
\end{longtable}
\end{footnotesize}
\end{center}

\textbf{Roles:} HQ=HQ\_ADMIN, ST=STATE\_MANAGER, SH=STORE\_MANAGER

\textbf{Access Types:}
\begin{itemize}[leftmargin=*]
    \item \textbf{F (Full)}: Complete access to all data across system
    \item \textbf{L (Filtered)}: Role-based data filtering (state/store scope)
    \item \textbf{R (Read-only)}: View access without modification rights
    \item \textbf{X (Denied)}: Access blocked (403 Forbidden)
\end{itemize}

\textbf{Key Points:}
\begin{itemize}[leftmargin=*]
    \item All endpoints use \texttt{/api/v2/} prefix (abbreviated for space)
    \item Filtering automatically applied based on user's assigned state/store
    \item Export functions inherit same permissions as data endpoints
    \item Peak hours restricted to management levels only
    \item Support email and time utilities available to all roles
\end{itemize}

\vfill

\begin{center}
{\small \textcopyright{} 2025 PizzaWorld Technical Documentation}
\end{center}

\end{document} 